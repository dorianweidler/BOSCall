\section{Fazit}
\label{sec:fazit}
Zunächst lässt sich als Fazit sagen, dass die App funktioniert und aufgrund des extremen Kostenvorteils bereits in Kürze in den Testbetrieb bei der Feuerwehr Kusel genommen wird. Somit lässt sich grundlegend schlussfolgern, dass das Projekt auf den ersten Blick erfolgreich war.

Die Probleme, die bei der Entwicklung auftraten, konnten wirksam behoben werden, dennoch sind einige unsaubere Lösungen im Code, die in späteren Iterationen entfernt werden sollen. Ein Beispiel wäre hier die Speicherung der Einheiten, die zurzeit noch als .json-Datei im Dateisystem erfolgt. Besser wäre es, wenn auch diese Funktionalität in Zukunft mittels Room gelöst wird.

Viel wichtiger als diese noch vorhandenen Notlösungen, die aber nach wie vor problemlos funktionieren, sind einige ungelöste Sicherheitsfragen. Beispielsweise sollten in naher Zukunft die Nachrichten die zwischen dem Alarmserver und dem Endgerät mittels Push ausgetauscht werden Ende-zu-Ende verschlüsselt werden. Dazu könnte man als Secret auch den API Key oder eben das Secret der Einheit nutzen, das im QR-Code bei der Registration hinterlegt ist. So kann einerseits sichergestellt werden, dass der Betreiber des Push Dienstes keinen Zugriff auf stark vertrauliche Daten (Einsatzdaten) hat, aber auch, dass die Daten bei einem Man-in-the-Middle Angriff nicht in falsche Hände gelangen. Alleine aus Gründen des Datenschutzes sollte das verhindert werden.

Ebenfalls sollten zukünftig alle Kommunikationen mit dem Backend nur noch auf Basis von SSL verschlüsselten HTTP Zugriffen (HTTPS) erfolgen. Da es sich hierbei aber zunächst um eine Probeanwendung handelt und auch noch keine SSL Zertifikate zur Verfügung standen, sollte zunächst die Kommunikation über HTTP erfolgen. Das ist bedeutend einfacher zu realisieren und für einen ersten Test auch völlig ausreichend. Für den späteren Produktivbetrieb sollte aber definitiv eine Umstellung auf HTTPS erfolgen.