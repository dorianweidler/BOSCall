\section{Einleitung}
Für Fördervereine der freiwilligen Feuerwehr ist es eine massive Kostenbelastung, wenn bereits wenige hundert Euro jedes Jahr zu bezahlen sind. Die Träger der Feuerwehren berufen sich bezüglich der Alarmierung per Smartphone darauf, dass bereits durch analoge oder digitale Meldeempfänger ausreichend vorgesorgt ist und die Einsatzkräfte adäquat alarmiert werden.

In der Feuerwehr Kusel wurde beispielsweise eine\\ firEmergency\cite{Alamos:FE2} Installation betrieben, die durch den Förderverein finanziert wurde. Das heißt Spenden, Mitgliederbeiträge und Veranstaltungserlöse werden genutzt um eine grundlegende Anforderung sicherzustellen. Ein Kostenvergleich zwischen einer aktuellen {firEmergency 2} Installation und der BOSCall Variante folgt in Kapitel~\ref{sec:kostenvergleich}.