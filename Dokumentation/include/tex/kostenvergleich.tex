\section{Kostenvergleich}
\label{sec:kostenvergleich}
Wie bereits einleitend erwähnt und auch technisch bedingt entstehen Kosten, um diese Art von digitalem Alarm zu realisieren. Zunächst soll hierbei auf eine fertige Lösung von einem etablierten Anbieter eingegangen werden. Diese ist bereits in der Feuerwehr Kusel, die als Testeinheit dient, in einer älteren, nicht mehr verfügbaren Version im Einsatz. Die App nennt sich APager (Pro), die Backend Software, die unabdingbar für die App ist, nennt sich firEmergency. Beides sind fertige Lösungen der Firma Alamos GmbH.

Betrachtet wird im Folgenden eine Lösung auf Basis einer serverseitigen Lizensierung. Das heißt die Kosten fallen auf Seiten des Betreibers an und sind entsprechend in Summe günstiger als der Kauf einzelner Lizenzen pro App. Die Preise stammen von \cite{Alamos:FE2Pricing}. Unterstützt werden soll eine Nutzerbasis von 200 Personen.
\begin{table}[]
	\centering
	\caption{Kostenaufstellung APager \& firEmergency}
	\label{tbl:FE2Pricing}
	\begin{tabular}{|l|r|}
		\hline
		{\ul \textbf{Beschreibung}}        & \multicolumn{1}{l|}{{\ul \textbf{Preis}}} \\ \hline
		firEmergency Paket 1 (30 Personen) & 179,99 €/Jahr                             \\ \hline
		+170 Personen                      & 593,30 €/Jahr                             \\ \hline
		Windows Office PC                  & ca. 400€                                  \\ \hline\hline
		\textbf{Summe}                  & \textbf{1173,29~€}                                  \\ \hline
	\end{tabular}
\end{table}

Bei der Aufstellung der Kosten in Tabelle~\ref{tbl:FE2Pricing} wurde berücksichtigt, dass Einzelpersonen günstiger sind als die größeren Pakete, wenn man wirklich nur die Personen und nicht weitere Features benötigt. Da BOSCall die anderen Features nicht bietet wird der Fairness halber mit dem preislich günstigsten Paket und den zusätzlichen Personen gerechnet. Ein Computer mit Windows Betriebssystem ist erforderlich um firEmergency zu betreiben.\cite{Alamos:FE2SystemRequirements} Aufsummiert fallen im ersten Jahr bereits 1173,29~€ an. In den Folgejahren, unter der Vorraussetzung die Preise ändern sich nicht und der Rechner wird einfach stetig weiterbetrieben, 773,29~€.

Das ist für einen kleinen, gemeinnützigen Verein kaum zu stemmen, insbesondere weil es sich nicht um einmalige sondern laufende Kosten handelt. Die Ersatz"-lösung, die im Rahmen dieser Veranstaltung entwickelt wurde, soll diese Kosten also drastisch unterschreiten, damit sich der zusätzliche Aufwand rechtfertigt. In Tabelle~\ref{tbl:BOSCallPricing} befindet sich eine Beispielaufstellung der Kosten, mit denen zu rechnen ist, um BOSCall zu betreiben. Da die Anwendung noch nicht in der Praxis erprobt ist, handelt es sich dabei zunächst um eine rein theoretische Aufstellung sämtlicher Kosten.

\begin{table}[]
	\centering
	\caption{Kostenaufstellung BOSCall}
	\label{tbl:BOSCallPricing}
	\begin{tabular}{|l|r|}
		\hline
		{\ul \textbf{Beschreibung}} & \multicolumn{1}{l|}{{\ul \textbf{Preis}}} \\ \hline
		Google Entwickler Zugang    & \$ 25                                     \\ \hline
		Virtueller Server           & 5 €/Monat                                 \\ \hline
		Raspberry Pi m. Zubehör     & 50 €                                      \\ \hline
		USB Soundkarte              & 6,25 €                                    \\ \hline \hline
		\textbf{Summe}                  & \textbf{137,75~€}                                  \\ \hline
	\end{tabular}
\end{table}

Man erkennt direkt, dass die Kosten erheblich geringer sind. Zudem verteilen sie sich fast nur auf einmalige Kosten. Der Google Entwickler Zugang ist erforderlich um Apps im Play Store zu veröffentlichen und sie so der breiten Masse an Feuerwehrmitgliedern der Feuerwehr Kusel einfach zugänglich zu machen. Der virtuelle Server ist für das Bereitstellen der Backend API erforderlich. Was der Dienst können muss wird in Kapitel~\ref{sec:funktionsweise} erläutert.  Der Raspberry Pi mit der USB Soundkarte ist für die lokale Installation eines Alarmauswerters erforderlich. Auch dabei wieder der Verweis auf Kapitel~\ref{sec:funktionsweise}. In Summe liegt man dabei im ersten Jahr bei ca. 137,75 €, abhängig vom Wechselkurs des Dollar. In den Folgejahren belaufen sich die Kosten auf 60 €. Dabei besteht zusätzlich der große Vorteil, dass man auch problemlos mehr Mitglieder anlegen kann ohne, dass sich die jährlichen Kosten erhöhen. Bereits der schwächste Linux Server, der heute üblicherweise angeboten wird, bietet erheblich mehr Leistung als erforderlich.